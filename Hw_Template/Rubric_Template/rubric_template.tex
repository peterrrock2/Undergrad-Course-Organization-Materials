\documentclass[letterpaper,11pt]{article}
\usepackage{amsmath, mathtools, comment, graphicx, fancyhdr, color, setspace, comment, multicol, hyperref, listings}
\usepackage[letterpaper,margin=1in,includehead=true]{geometry}
\usepackage{amssymb,MnSymbol}
\usepackage{amsthm}


\newif\ifprintanswers
  
\newenvironment{solution}
  {\flushleft\color{blue}\textbf{Solution:}}
  {}



\begin{document}

\noindent No need to put the question statement here. Just say what problem this is for (e.g. additional problem 1)

\begin{solution} Put the full solution here. Maybe with an equation
    \[a^2 + b^2 = c^2\]
Please do not define custom commands. So to make $\mathbb{R}$ you need to type \texttt{\textbackslash mathbb\{R\}}.

%======================
%    RUBRIC
%======================

\textbf{Rubric:}
\begin{table}[h!]
    \centering
    \color{blue}
    \begin{tabular}{|p{0.25in}|p{5in}|}
        \hline
        0 & No Attempt. \\ \hline
        1 & Attempt made. You may or may not put things here \\ \hline
        2 & Mostly there. Say what distinguishes this from a 1. What is the ``leap'' in the problem? \\ \hline
        3 & Correct. Say what distinguishes this from a 2, and what sort of mistakes are allowed to still receive full credit. \\ \hline
    \end{tabular}
\end{table}
\end{solution}
\end{document}