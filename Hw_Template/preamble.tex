% Template Author: Peter Rock
% Last Modified 29 December 2022

% This file makes use of a number of macros and some randomizer code
% to make modifications to future assignments easier. If the tex is not
% compiling, check to make sure that the commands \pathToNonTextbook and
% \pathToTextbook are correct at the bottom of the preamble



\documentclass[letterpaper,11pt]{article}
\usepackage{amsmath, mathtools, comment, graphicx, fancyhdr, color, setspace, comment, multicol, hyperref}
\usepackage[letterpaper,margin=1in,includehead=true]{geometry}
\pagestyle{fancy}
\usepackage{enumitem}
\usepackage{tikz,tikz-3dplot,pgfplots}
\pgfplotsset{compat=1.18}
\usepackage{soul}
\usepackage{amssymb,MnSymbol}
\usepackage{amsthm}
\usepackage{tcolorbox}
\usepackage{latexsym}
\usepackage{tikz-cd}
\usepackage{mathtools}
\usepackage[none]{hyphenat}
\usepackage{xcolor}
\usepackage{comment}


\setlength{\headheight}{15pt}%needed to remove fancyhdr error--not crucial%


\renewcommand\vec{\mathbf}


% ======================================================================
%                  CODE FOR RANDOM QUESTION SELECTION
% ======================================================================
%enable the latex3 coding language
\ExplSyntaxOn

\seq_new:N \l__me_random_seq
\int_new:N \l__me_items_int
\int_new:N \l__me_length_int

\msg_new:nnn { me } { range-too-small }
  { range~ [#2,~#3]~ too~ small~ for~ random~ list~ with~ #1~ elements }
\msg_new:nnn { me } { unknown-seq } { no~ seq~ with~ name~ #1~ defined }

\cs_new_protected:Npn \__me_set_random_seq:nnn #1#2#3
  {
    % #1: number of subset elements
    % #2: start of range
    % #3: end of range
    \int_set:Nn \l__me_items_int {#1}
    \int_set:Nn \l__me_length_int { #3 - (#2) + \c_one_int }
    \seq_clear:N \l__me_random_seq
    \int_compare:nNnT \l__me_items_int > \l__me_length_int
      {
        \msg_error:nnxxx { me } { range-too-small }
          { \int_eval:n{#1} }
          { \int_eval:n{#2} }
          { \int_eval:n{#3} }
      }
    % build a sorted sequence with the full range
    \int_step_inline:nnn {#2} {#3}
      { \seq_put_right:Nn \l__me_random_seq {##1} }
    % remove the extra items (randomly)
    \int_step_inline:nnnn
      \l__me_length_int
      { -1 }
      { \l__me_items_int + \c_one_int }
      {
        \seq_pop_item:NnN
          \l__me_random_seq { \int_rand:n {##1} } \l_tmpa_tl
      }
    % result is sorted
  }

\cs_new_protected:Npn \me_loop_random_seq:nnnn #1#2#3#4
  {
    % #1: number of subset elements
    % #2: start of range
    % #3: end of range
    % #4: code to do on each element
    \__me_set_random_seq:nnn {#1} {#2} {#3}
    \seq_map_inline:Nn \l__me_random_seq {#4}
  }
\NewDocumentCommand \loopoverrandomsubsetofrange { m O{1} m +m }
  { \me_loop_random_seq:nnnn {#1} {#2} {#3} {#4} }

\cs_new_protected:Npn \me_set_named_random_seq:nnnn #1#2#3#4
  {
    % #1: name
    % #2: number of subset elements
    % #3: start of range
    % #4: end of range
    \__me_set_random_seq:nnn {#2} {#3} {#4}
    \seq_clear_new:c { l__me_ #1 _random_seq }
    \seq_set_eq:cN { l__me_ #1 _random_seq } \l__me_random_seq
  }
\NewDocumentCommand \setrandomsubsetofrange { m m O{1} m }
  { \me_set_named_random_seq:nnnn {#1} {#2} {#3} {#4} }

\cs_new_protected:Npn \me_loop_named_random_seq:nn #1#2
  {
    \seq_if_exist:cTF { l__me_ #1 _random_seq }
      { \seq_map_inline:cn { l__me_ #1 _random_seq } {#2} }
      { \msg_error:nnn { me } { unknown-seq } {#1} }
  }
\NewDocumentCommand \userandomsubsetofrange { m +m }
  { \me_loop_named_random_seq:nn {#1} {#2} }

\ExplSyntaxOff


\newcommand\randomincludefile[2]
{%
% #1: list name
% #2: name before numbering system
\userandomsubsetofrange{#1}{\input{#2##1.tex}}%
}




% ======================================================================
%                  SETUP FOR THE PROBLEMS
% ======================================================================


%paths are taken relative to the main file
\newcommand{\pathToNonTextbook}{Problems/Non\_Textbook/non_text_problem}
\newcommand{\pathToTextbook}{Problems/Textbook/text_problem}
\newcommand{\pathToAdditional}{Problems/Additional_Problems/add_problem}

\newcounter{numNonTextbookFiles}
\newcounter{numTextbookFiles}
\newcounter{numAddFiles}



%the next few lines count the number of available non-textbook problems
\setcounter{numNonTextbookFiles}{1}

\newboolean{stopNT}
\whiledo{\NOT\boolean{stopNT}}{
  \stepcounter{numNonTextbookFiles}
  \IfFileExists{\pathToNonTextbook\thenumNonTextbookFiles.tex}{}{
    \addtocounter{numNonTextbookFiles}{-1}
    \setboolean{stopNT}{true}
  }
}


%the next few lines count the number of available textbook problems
\setcounter{numTextbookFiles}{1}

\newboolean{stopT}
\whiledo{\NOT\boolean{stopT}}{
  \stepcounter{numTextbookFiles}
  \IfFileExists{\pathToTextbook\thenumTextbookFiles.tex}{}{
    \addtocounter{numTextbookFiles}{-1}
    \setboolean{stopT}{true}
  }
}


%the next few lines count the number of available additional problems
\setcounter{numAddFiles}{1}

\newboolean{stopA}
\whiledo{\NOT\boolean{stopA}}{
  \stepcounter{numAddFiles}
  \IfFileExists{\pathToAdditional\thenumAddFiles.tex}{}{
    \addtocounter{numAddFiles}{-1}
    \setboolean{stopA}{true}
  }
}

%counter to keep track of the number of problems 
%only used so the enumerate environments behave
\newcounter{enumcount}

% ======================================================================
%                  THINGS TO HIDE AND SHOW SOLUTIONS
% ======================================================================

\newif\ifprintanswers
  
\newenvironment{solution}
  {\color{blue}\textbf{Solution:}}
  {}
